\documentclass[12pt, class=article, crop=false]{standalone}
\usepackage[subpreambles=true]{standalone}

% general use packages
\usepackage{import,
            graphicx,
            parskip,
            url,
            amsmath,
            wrapfig,
            soul}

% box setup
\usepackage[most]{tcolorbox}
\tcbuselibrary{breakable}

% margin setup
\usepackage[top=2cm,
            bottom=2cm,
            left=2cm,
            right=2cm]{geometry}%set margin

% maintext font
\usepackage[T1]{fontenc}
\usepackage{tgtermes}

% side caption figure
\usepackage{sidecap}
\sidecaptionvpos{figure}{t}

% citation style
\usepackage[sort&compress]{natbib}
\setcitestyle{square}
\setcitestyle{comma}
\bibliographystyle{bibstyle}

% caption setup
\usepackage[font={fontenc, small}, labelfont={bf, small}]{caption}

% suppress page numbers
\pagenumbering{gobble}

\begin{document}

\section*{Specific Aims}

Gut microbiomes are crucial in immune system development and pathogen defense.
Mounting evidence suggests that the abnormal state of gut microbial communities, termed ``dysbiosis,'' is common in patients with inflammatory bowel diseases and obesity.
These studies often view dysbiosis as a result of undesired gut environments, emphasizing the role of ecological niches available for beneficial and/or harmful taxa.
Ecological theory predicts, however, that the structure of gut microbial communities can be sensitive to the order of species arrival, potentially leading to dysbiosis even when the optimal environment for microbial species characteristic of a healthy individual is maintained.
Understanding the assembly of microbial species is therefore vital for informing clinical treatments effectively.
However, quantifying the strength of priority effects is challenging.
It requires the experimental manipulation of species arrival in simplified communities with a few species, limiting our ability to understand the behavior and assembly processes of species-rich microbial communities.
In addition, \textit{in vivo} experiments are infeasible in many cases for logistical and ethical reasons, especially when the focus is on the human microbiome.
These challenges with experimental approaches create a need for the use of observational data to study priority effects, but knowing the arrival order of microbial species in non-experimental systems is virtually impossible.
Resolving this fundamental problem requires the development of a statistical method that can infer the strength of priority effects \textit{without knowing the history of species arrival.}
In this proposed research, \ul{we will develop and demonstrate the utility of a new statistical method capable of quantifying priority effects from observational data.}
The method will enable us to address questions concerning the prevalence and contributing factors of priority effects, with implications for enhancing the effectiveness of clinical treatments.
Our proposal constitutes three specific aims outlined below.

\textbf{Specific Aim 1: Development of a New Modeling Approach} --
In this aim, we will establish a theoretical foundation for a statistical method to detect communities sensitive to priority effects.
A pivotal initial step involves acquiring reliable competition estimates from empirical time series. However, practical challenges arise in directly estimating these parameters, especially in hyper-diverse microbial communities, due to the ``curse of dimensionality.''
That is, as the number of species increases, the time series length required for estimation grows exponentially.
We aim to tackle this challenge by developing a simple approach to reduce the parameter dimension of the statistical model.
The effectiveness of our proposed method will be validated through simulation experiments.

\textbf{Specific Aim 2: Experimental Validation} --
While simulation experiments in Specific Aim 1 represent the first step to confirm the credibility of our method, it is imperative to validate its ability to predict priority effects when applied to real biological data.
In this aim, we will evaluate the performance of the proposed method using experimental data extracted from published studies, where the order of species introduction was experimentally manipulated.
The experimental manipulation of species introduction allows us to experimentally estimate the strength of priority effects.
We will assess whether our method can predict the experimental estimate of priority effects without the information for the order of species introduction.

\textbf{Specific Aim 3: Application to Observational Data} --
This aim involves applying our method to unmanipulated gut systems to explore when priority effects are strong in the human gut microbiota.
We will curate and analyze published time-series data of human gut microbiota, posing several key questions:
(i) Does the strength of priority effects vary among geographic regions?
(ii) Do Western-style diets influence the strength of priority effects?
(iii) Are priority effects stronger at younger ages?
(iv) Are priority effects stronger in individuals with unhealthy states?

\end{document}