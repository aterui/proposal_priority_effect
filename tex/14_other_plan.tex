\documentclass[12pt, class=article, crop=false]{standalone}
\usepackage[subpreambles=true]{standalone}

% general use packages
\usepackage{import,
            graphicx,
            parskip,
            url,
            amsmath,
            wrapfig,
            soul}

% box setup
\usepackage[most]{tcolorbox}
\tcbuselibrary{breakable}

% margin setup
\usepackage[top=2cm,
            bottom=2cm,
            left=2cm,
            right=2cm]{geometry}%set margin

% maintext font
\usepackage[T1]{fontenc}
\usepackage{tgtermes}

% side caption figure
\usepackage{sidecap}
\sidecaptionvpos{figure}{t}

% citation style
\usepackage[sort&compress]{natbib}
\setcitestyle{square}
\setcitestyle{comma}
\bibliographystyle{bibstyle}

% caption setup
\usepackage[font={fontenc, small}, labelfont={bf, small}]{caption}

\begin{document}

\section*{Other Plan(s)}

\subsubsection*{Data/Code Management}

\textbf{Data}: While we do not collect data, we will use experimental/observational data available from public data repositories and/or publications.
Data and metadata content and format will adhere to established standards, ensuring that the data are Findable, Accessible, Interoperable, and Reproducible (FAIR).

Throughout the project, data will exist in three forms.
\textit{Raw data}: This data is acquired immediately after collection from data repositories/pubcations, prior to quality control and assurance checks.
\textit{Intermediate data}: Raw data that has undergone quality assurance checks.
It lists all raw data and flags data that did not pass quality checks, providing justification for exclusion.
\textit{Final data}: This is a subset of raw data considered of high enough quality for analysis.

The original data sourced from data repositories will be saved as spreadsheet (.csv) or R files (.rds) with ``raw'' in the filename.
Metadata from corresponding data repositories or through communications with data providers will be saved as a "README" markdown file (.md) in the same repository, allowing users to seamlessly access that information as needed.
All data will be hosted in Github along with source codes for analysis so readers can reproduce research findings.
We will track any changes to data using Github's version control system.

\textbf{Source Code}: We will develop and store our source codes for data formatting and statistical analysis in either R or Python scripts.
These scripts will be hosted on GitHub, ensuring version control and collaboration among project team members. 
GitHub's version control system will enable us to track changes, collaborate seamlessly, and maintain a historical record of code modifications throughout the project's duration.
To ensure that our research is accessible to the wider community, we commit to sharing our source codes as open-source. These codes will be made available under permissive licenses such as MIT or BSD. This allows others to use, modify, and build upon our work.

\subsubsection*{Policy for Product Access and Sharing}

\textbf{Repositories}: All project products will be publicly accessible through either Zenodo.

Zenodo, a public repository that issues DOIs (Digital Object Identifiers) (https://zenodo.org/), provides a stable and citable location for data.
It also seamlessly integrates with GitHub, facilitating easy access to linked source code.
Additionally, Zenodo offers a version control system, enabling us to effectively manage any potential updates or modifications to published datasets as needed. Zenodo boasts a rich history of utilization in ecology and various other scientific domains.

\textbf{Publication and Timing}: We will share our data and source codes following the publication of major findings from our project or, at the latest, within two years after the project's completion, whichever comes first.
This timeline ensures that our work remains accessible to the public and can benefit the research community promptly.

We acknowledge that most of the datasets we will utilize are publicly available.
However, for any data that is author-owned/confidential and cannot be shared with the public, we will adhere to ethical and legal guidelines.
We will provide references to the appropriate contacts, data archives, or publications that can guide readers on how to access or request the restricted data.

By adhering to these data management and sharing practices, our project aims to promote transparency, reproducibility, and open access to our computational resources and data.
This approach ensures that our research can be leveraged by the broader scientific community and facilitates the validation and extension of our findings.

\end{document}