\documentclass[12pt, class=article, crop=false]{standalone}
\usepackage[subpreambles=true]{standalone}

% general use packages
\usepackage{import,
            graphicx,
            parskip,
            url,
            amsmath,
            wrapfig,
            soul}

% box setup
\usepackage[most]{tcolorbox}
\tcbuselibrary{breakable}

% margin setup
\usepackage[top=2cm,
            bottom=2cm,
            left=2cm,
            right=2cm]{geometry}%set margin

% maintext font
\usepackage[T1]{fontenc}
\usepackage{tgtermes}

% side caption figure
\usepackage{sidecap}
\sidecaptionvpos{figure}{t}

% citation style
\usepackage[sort&compress]{natbib}
\setcitestyle{square}
\setcitestyle{comma}
\bibliographystyle{bibstyle}

% caption setup
\usepackage[font={fontenc, small}, labelfont={bf, small}]{caption}

% suppress page numbers
\pagenumbering{gobble}

\begin{document}

\section*{Project Narrative}

The health consequences of dysbiosis have stimulated researchers to study how microbial communities are shaped in the gut, with a persistent debate about when species' arrival orders have strong influences on the community assembly (priority effects).
In this proposed project, we will develop a new statistical method that identifies microbial communities sensitive to priority effects, aiming to provide insights into the conditions under which priority effects strongly affect the human microbiome.
Successful completion of this project will furnish a crucial statistical resource, offering the ability to inform when priority effects are strong.
\end{document}